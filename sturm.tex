\documentclass{article}

%% Created with wxMaxima 15.08.2

\setlength{\parskip}{\medskipamount}
\setlength{\parindent}{0pt}
\usepackage[utf8]{inputenc}
\DeclareUnicodeCharacter{00B5}{\ensuremath{\mu}}
\usepackage{graphicx}
\usepackage{color}
\usepackage{amsmath}
\usepackage{ifthen}
\usepackage{amssymb}
\usepackage{amsmath}
\newsavebox{\picturebox}
\newlength{\pictureboxwidth}
\newlength{\pictureboxheight}
\newcommand{\includeimage}[1]{
    \savebox{\picturebox}{\includegraphics{#1}}
    \settoheight{\pictureboxheight}{\usebox{\picturebox}}
    \settowidth{\pictureboxwidth}{\usebox{\picturebox}}
    \ifthenelse{\lengthtest{\pictureboxwidth > .95\linewidth}}
    {
        \includegraphics[width=.95\linewidth,height=.80\textheight,keepaspectratio]{#1}
    }
    {
        \ifthenelse{\lengthtest{\pictureboxheight>.80\textheight}}
        {
            \includegraphics[width=.95\linewidth,height=.80\textheight,keepaspectratio]{#1}
            
        }
        {
            \includegraphics{#1}
        }
    }
}
\DeclareMathOperator{\abs}{abs}
\usepackage{animate} % This package is required because the wxMaxima configuration option
                      % "Export animations to TeX" was enabled when this file was generated.

\definecolor{labelcolor}{RGB}{100,0,0}

\begin{document}

\pagebreak{}
{\Huge {\sc Sucesión de Sturm}}
\setcounter{section}{0}
\setcounter{subsection}{0}
\setcounter{figure}{0}


Asignamos a s(x) el polinomio a partir del cual queremos obtener la sucesión de Sturm.


\noindent
%%%%%%%%%%%%%%%
%%% INPUT:
\begin{minipage}[t]{8ex}\color{red}\bf
\begin{verbatim}
-->  
\end{verbatim}
\end{minipage}
\begin{minipage}[t]{\textwidth}\color{blue}
\begin{verbatim}
 s(x) := x^4 + 2*x^3 -3*x^2 -4*x-1;
\end{verbatim}
\end{minipage}
%%% OUTPUT:

\[\displaystyle
\parbox{10ex}{$\color{labelcolor}\mathrm{\tt (\%o187) }\quad $}
\mathrm{s}\left( x\right) :=-1-4\cdot x-3\cdot {{x}^{2}}+2\cdot {{x}^{3}}+{{x}^{4}}\mbox{}
\]
%%%%%%%%%%%%%%%


\noindent
%%%%%%%%%%%%%%%
%%% INPUT:
\begin{minipage}[t]{8ex}\color{red}\bf
\begin{verbatim}
-->  
\end{verbatim}
\end{minipage}
\begin{minipage}[t]{\textwidth}\color{blue}
\begin{verbatim}
p(x) := s(x);
\end{verbatim}
\end{minipage}
%%% OUTPUT:

\[\displaystyle
\parbox{10ex}{$\color{labelcolor}\mathrm{\tt (\%o188) }\quad $}
\mathrm{p}\left( x\right) :=\mathrm{s}\left( x\right) \mbox{}
\]
%%%%%%%%%%%%%%%

Definimos el primero término de la sucesión.


\noindent
%%%%%%%%%%%%%%%
%%% INPUT:
\begin{minipage}[t]{8ex}\color{red}\bf
\begin{verbatim}
-->  
\end{verbatim}
\end{minipage}
\begin{minipage}[t]{\textwidth}\color{blue}
\begin{verbatim}
 f1(x) := s(x);
\end{verbatim}
\end{minipage}
%%% OUTPUT:

\[\displaystyle
\parbox{10ex}{$\color{labelcolor}\mathrm{\tt (\%o189) }\quad $}
\mathrm{f1}\left( x\right) :=\mathrm{s}\left( x\right) \mbox{}
\]
%%%%%%%%%%%%%%%

Definimos el segundo elemento de la sucesión derivando el anterior.


\noindent
%%%%%%%%%%%%%%%
%%% INPUT:
\begin{minipage}[t]{8ex}\color{red}\bf
\begin{verbatim}
-->  
\end{verbatim}
\end{minipage}
\begin{minipage}[t]{\textwidth}\color{blue}
\begin{verbatim}
define(f2(x),diff(s(x), x, 1));
\end{verbatim}
\end{minipage}
%%% OUTPUT:

\[\displaystyle
\parbox{10ex}{$\color{labelcolor}\mathrm{\tt (\%o190) }\quad $}
\mathrm{f2}\left( x\right) :=4\cdot {{x}^{3}}+6\cdot {{x}^{2}}-6\cdot x-4\mbox{}
\]
%%%%%%%%%%%%%%%

Creamos la lista que va a tener todos los elementos de la sucesión.


\noindent
%%%%%%%%%%%%%%%
%%% INPUT:
\begin{minipage}[t]{8ex}\color{red}\bf
\begin{verbatim}
-->  
\end{verbatim}
\end{minipage}
\begin{minipage}[t]{\textwidth}\color{blue}
\begin{verbatim}
f(x):=[f1(x),f2(x)];
\end{verbatim}
\end{minipage}
%%% OUTPUT:

\[\displaystyle
\parbox{10ex}{$\color{labelcolor}\mathrm{\tt (\%o191) }\quad $}
\mathrm{f}\left( x\right) :=[\mathrm{f1}\left( x\right) ,\mathrm{f2}\left( x\right) ]\mbox{}
\]
%%%%%%%%%%%%%%%


\noindent
%%%%%%%%%%%%%%%
%%% INPUT:
\begin{minipage}[t]{8ex}\color{red}\bf
\begin{verbatim}
-->  
\end{verbatim}
\end{minipage}
\begin{minipage}[t]{\textwidth}\color{blue}
\begin{verbatim}
f(x);
\end{verbatim}
\end{minipage}
%%% OUTPUT:

\[\displaystyle
\parbox{10ex}{$\color{labelcolor}\mathrm{\tt (\%o192) }\quad $}
[{{x}^{4}}+2\cdot {{x}^{3}}-3\cdot {{x}^{2}}-4\cdot x-1,4\cdot {{x}^{3}}+6\cdot {{x}^{2}}-6\cdot x-4]\mbox{}
\]
%%%%%%%%%%%%%%%

Obtenemos el término i de la sucesión dividiendo el término i-2 entre el i-1. El término i será el resto cambiado de signo.


\noindent
%%%%%%%%%%%%%%%
%%% INPUT:
\begin{minipage}[t]{8ex}\color{red}\bf
\begin{verbatim}
-->  
\end{verbatim}
\end{minipage}
\begin{minipage}[t]{\textwidth}\color{blue}
\begin{verbatim}
 for i:3 while (hipow(p(x),x) > 0) do(
    define(p(x),-divide(f(x)[i-2],f(x)[i-1])[2]),
    define(f(x), append(f(x),[p(x)]))
);
\end{verbatim}
\end{minipage}
%%% OUTPUT:

\[\displaystyle
\parbox{10ex}{$\color{labelcolor}\mathrm{\tt (\%o193) }\quad $}
\mathit{done}\mbox{}
\]
%%%%%%%%%%%%%%%

Se muestra la lista con los términos de la sucesión.


\noindent
%%%%%%%%%%%%%%%
%%% INPUT:
\begin{minipage}[t]{8ex}\color{red}\bf
\begin{verbatim}
-->  
\end{verbatim}
\end{minipage}
\begin{minipage}[t]{\textwidth}\color{blue}
\begin{verbatim}
f(x);
\end{verbatim}
\end{minipage}
%%% OUTPUT:

\[\displaystyle
\parbox{10ex}{$\color{labelcolor}\mathrm{\tt (\%o194) }\quad $}
[{{x}^{4}}+2\cdot {{x}^{3}}-3\cdot {{x}^{2}}-4\cdot x-1,4\cdot {{x}^{3}}+6\cdot {{x}^{2}}-6\cdot x-4,\frac{2+9\cdot x+9\cdot {{x}^{2}}}{4},\frac{40+80\cdot x}{9},\frac{1}{16}]\mbox{}
\]
%%%%%%%%%%%%%%%

num contendrá el valor donde se evaluará los términos de la sucesión.


\noindent
%%%%%%%%%%%%%%%
%%% INPUT:
\begin{minipage}[t]{8ex}\color{red}\bf
\begin{verbatim}
-->  
\end{verbatim}
\end{minipage}
\begin{minipage}[t]{\textwidth}\color{blue}
\begin{verbatim}
num : -5;
\end{verbatim}
\end{minipage}
%%% OUTPUT:

\[\displaystyle
\parbox{10ex}{$\color{labelcolor}\mathrm{\tt (\%o197) }\quad $}
-5\mbox{}
\]
%%%%%%%%%%%%%%%

Se contabiliza el número de cambios de signos que se producen en la sucesión evaluada en num y se guarda en cont.


\noindent
%%%%%%%%%%%%%%%
%%% INPUT:
\begin{minipage}[t]{8ex}\color{red}\bf
\begin{verbatim}
-->  
\end{verbatim}
\end{minipage}
\begin{minipage}[t]{\textwidth}\color{blue}
\begin{verbatim}
cont : 0;
\end{verbatim}
\end{minipage}
%%% OUTPUT:

\[\displaystyle
\parbox{10ex}{$\color{labelcolor}\mathrm{\tt (\%o198) }\quad $}
0\mbox{}
\]
%%%%%%%%%%%%%%%


\noindent
%%%%%%%%%%%%%%%
%%% INPUT:
\begin{minipage}[t]{8ex}\color{red}\bf
\begin{verbatim}
-->  
\end{verbatim}
\end{minipage}
\begin{minipage}[t]{\textwidth}\color{blue}
\begin{verbatim}
for i:1 thru (length(f(x))-1) do(
    if f(num)[i]*f(num)[i+1] < 0 then (
        cont : cont + 1
    )
);
\end{verbatim}
\end{minipage}
%%% OUTPUT:

\[\displaystyle
\parbox{10ex}{$\color{labelcolor}\mathrm{\tt (\%o200) }\quad $}
\mathit{done}\mbox{}
\]
%%%%%%%%%%%%%%%

Muestra cont.


\noindent
%%%%%%%%%%%%%%%
%%% INPUT:
\begin{minipage}[t]{8ex}\color{red}\bf
\begin{verbatim}
-->  
\end{verbatim}
\end{minipage}
\begin{minipage}[t]{\textwidth}\color{blue}
\begin{verbatim}
cont;
\end{verbatim}
\end{minipage}
%%% OUTPUT:

\[\displaystyle
\parbox{10ex}{$\color{labelcolor}\mathrm{\tt (\%o201) }\quad $}
4\mbox{}
\]
%%%%%%%%%%%%%%%
\newpage

%%%%%%%%%%%%%%%%%%%%%%%%%%

%	EJERCICIO 14

%%%%%%%%%%%%%%%%%%%%%%%%%%
{\Huge {\sc Ejercicio 14}}\\

Considera el polinomio $p(x) = 2x^5 - x^4 - 4x^3 + 2x^2 - 6x + 3$.\\
a) Calcula una sucesión de Sturm asociada a $p(x)$.\\

$$f_0(x) = 2\cdot {{x}^{5}}-{{x}^{4}}+4\cdot {{x}^{3}}+2\cdot {{x}^{2}}-6\cdot x+3$$\\
$$f_1(x) = 10\cdot {{x}^{4}}-4\cdot {{x}^{3}}+12\cdot {{x}^{2}}+4\cdot x-6$$\\
$$f_2(x) = -\frac{72-118\cdot x+36\cdot {{x}^{2}}+38\cdot {{x}^{3}}}{25}$$\\
$$f_3(x) = -\frac{7050-20500\cdot x+20150\cdot {{x}^{2}}}{361}$$\\
$$f_4(x) = -\frac{8987456\cdot x-7451040}{4060225}$$\\
$$f_5(x) = \frac{1181525475}{109253762}$$\\
\\

b) Halla una cota superior e inferior de las raíces de $p(x)$.\\

c) Localiza todas las raíces reales de $p(x)$ en un intervalo cada una.\\

\begin{table}[h]
\centering
\label{my-label}
\begin{tabular}{|l|l|l|}
\hline
                            & -4 & 4 \\ \hline
Número de  cambios de signo & 4  & 1 \\ \hline
\end{tabular}
\end{table}

\begin{table}[h]
\centering
\label{my-label}
\begin{tabular}{|l|l|l|l|}
\hline
                            & -4 & 0 & 4 \\ \hline
Número de  cambios de signo & 4  & 3 & 1 \\ \hline
\end{tabular}
\end{table}

\begin{table}[h]
\centering
\label{my-label}
\begin{tabular}{|l|l|l|l|}
\hline
                            & 0 & 2 & 4 \\ \hline
Número de  cambios de signo & 3 & 1 & 1 \\ \hline
\end{tabular}
\end{table}

\begin{table}[]
\centering
\label{my-label}
\begin{tabular}{|l|l|l|l|}
\hline
                            & 0 & 1 & 2 \\ \hline
Número de  cambios de signo & 3 & 2 & 1 \\ \hline
\end{tabular}
\end{table}

%%%%%%%%%%%%%%%%%%%%%%%%%%

%	EJERCICIO 18

%%%%%%%%%%%%%%%%%%%%%%%%%%
{\Huge {\sc Ejercicio 18}}\\

Considera el sistema de ecuaciones:
$$g(x) = \left\{
\begin{array}{c l}
 3x_1 - cos(x_2x_3) - \frac{1}{2}        &= 0\\
 x_1^2 - 81(x_2+0.1)^2 + sin(x_3) + 1.06 &= 0\\
 e^{-x_1x_2} + 20x_3 + \frac{10\pi-3}{3} &= 0
\end{array}
\right.
$$

a) Escribe el sistema anterior en la forma $x = g(x)$ despejando en la ecuación $i$ la variable $x_i$ , $i = 1, 2, 3$.

$$g(x) = \begin{pmatrix}
    x_1 \\
    x_2 \\
    x_3     
\end{pmatrix}
=\begin{pmatrix}
    \frac{cos(x_2x_3)+\frac{1}{2}}{3} \\
    \sqrt{\frac{sin(x_3)+x_1^2+1.06}{81}}-0.1 \\
    \frac{e^{-x_1x_2}+ \frac{10\pi-3}{3}}{-20}     
\end{pmatrix}$$

b) Demuestra, utilizando el resultado del ejercicio anterior que el sistema de ecuaciones tiene una única solución en
$$D = \{(x_1 , x_2 , x_3) \in \mathbb{R}^3 | 1 \le x_i \le 1, i = 1, 2, 3\}$$

\noindent
%%%%%%%%%%%%%%%
%%% INPUT:
\begin{minipage}[t]{8ex}\color{red}\bf
\begin{verbatim}
-->  
\end{verbatim}
\end{minipage}
\begin{minipage}[t]{\textwidth}\color{blue}
\begin{verbatim}
diff(g1(x,y,z),x);
\end{verbatim}
\end{minipage}
%%% OUTPUT:

\[\displaystyle
\parbox{10ex}{$\color{labelcolor}\mathrm{\tt (\%o33) }\quad $}
0\mbox{}
\]
%%%%%%%%%%%%%%%


\noindent
%%%%%%%%%%%%%%%
%%% INPUT:
\begin{minipage}[t]{8ex}\color{red}\bf
\begin{verbatim}
-->  
\end{verbatim}
\end{minipage}
\begin{minipage}[t]{\textwidth}\color{blue}
\begin{verbatim}
diff(g1(x,y,z),y);
\end{verbatim}
\end{minipage}
%%% OUTPUT:

\[\displaystyle
\parbox{10ex}{$\color{labelcolor}\mathrm{\tt (\%o34) }\quad $}
-\frac{z\cdot \mathrm{sin}\left( y\cdot z\right) }{3}\mbox{}
\]
%%%%%%%%%%%%%%%


\noindent
%%%%%%%%%%%%%%%
%%% INPUT:
\begin{minipage}[t]{8ex}\color{red}\bf
\begin{verbatim}
-->  
\end{verbatim}
\end{minipage}
\begin{minipage}[t]{\textwidth}\color{blue}
\begin{verbatim}
diff(g1(x,y,z),z);
\end{verbatim}
\end{minipage}
%%% OUTPUT:

\[\displaystyle
\parbox{10ex}{$\color{labelcolor}\mathrm{\tt (\%o35) }\quad $}
-\frac{y\cdot \mathrm{sin}\left( y\cdot z\right) }{3}\mbox{}
\]
%%%%%%%%%%%%%%%


\noindent
%%%%%%%%%%%%%%%
%%% INPUT:
\begin{minipage}[t]{8ex}\color{red}\bf
\begin{verbatim}
-->  
\end{verbatim}
\end{minipage}
\begin{minipage}[t]{\textwidth}\color{blue}
\begin{verbatim}
diff(g2(x,y,z),x);
\end{verbatim}
\end{minipage}
%%% OUTPUT:

\[\displaystyle
\parbox{10ex}{$\color{labelcolor}\mathrm{\tt (\%o36) }\quad $}
\frac{x}{9\cdot \sqrt{\mathrm{sin}\left( z\right) +{{x}^{2}}+1.06}}\mbox{}
\]
%%%%%%%%%%%%%%%


\noindent
%%%%%%%%%%%%%%%
%%% INPUT:
\begin{minipage}[t]{8ex}\color{red}\bf
\begin{verbatim}
-->  
\end{verbatim}
\end{minipage}
\begin{minipage}[t]{\textwidth}\color{blue}
\begin{verbatim}
diff(g2(x,y,z),y);
\end{verbatim}
\end{minipage}
%%% OUTPUT:

\[\displaystyle
\parbox{10ex}{$\color{labelcolor}\mathrm{\tt (\%o37) }\quad $}
0\mbox{}
\]
%%%%%%%%%%%%%%%


\noindent
%%%%%%%%%%%%%%%
%%% INPUT:
\begin{minipage}[t]{8ex}\color{red}\bf
\begin{verbatim}
-->  
\end{verbatim}
\end{minipage}
\begin{minipage}[t]{\textwidth}\color{blue}
\begin{verbatim}
diff(g2(x,y,z),z);
\end{verbatim}
\end{minipage}
%%% OUTPUT:

\[\displaystyle
\parbox{10ex}{$\color{labelcolor}\mathrm{\tt (\%o38) }\quad $}
\frac{\mathrm{cos}\left( z\right) }{18\cdot \sqrt{\mathrm{sin}\left( z\right) +{{x}^{2}}+1.06}}\mbox{}
\]
%%%%%%%%%%%%%%%


\noindent
%%%%%%%%%%%%%%%
%%% INPUT:
\begin{minipage}[t]{8ex}\color{red}\bf
\begin{verbatim}
-->  
\end{verbatim}
\end{minipage}
\begin{minipage}[t]{\textwidth}\color{blue}
\begin{verbatim}
diff(g3(x,y,z),x);
\end{verbatim}
\end{minipage}
%%% OUTPUT:

\[\displaystyle
\parbox{10ex}{$\color{labelcolor}\mathrm{\tt (\%o39) }\quad $}
\frac{\mathrm{log}\left( e\right) \cdot y}{20\cdot {{e}^{x\cdot y}}}\mbox{}
\]
%%%%%%%%%%%%%%%


\noindent
%%%%%%%%%%%%%%%
%%% INPUT:
\begin{minipage}[t]{8ex}\color{red}\bf
\begin{verbatim}
-->  
\end{verbatim}
\end{minipage}
\begin{minipage}[t]{\textwidth}\color{blue}
\begin{verbatim}
diff(g3(x,y,z),y);
\end{verbatim}
\end{minipage}
%%% OUTPUT:

\[\displaystyle
\parbox{10ex}{$\color{labelcolor}\mathrm{\tt (\%o40) }\quad $}
\frac{\mathrm{log}\left( e\right) \cdot x}{20\cdot {{e}^{x\cdot y}}}\mbox{}
\]
%%%%%%%%%%%%%%%

\noindent
%%%%%%%%%%%%%%%
%%% INPUT:
\begin{minipage}[t]{8ex}\color{red}\bf
\begin{verbatim}
-->  
\end{verbatim}
\end{minipage}
\begin{minipage}[t]{\textwidth}\color{blue}
\begin{verbatim}
diff(g3(x,y,z),z);
\end{verbatim}
\end{minipage}
%%% OUTPUT:

\[\displaystyle
\parbox{10ex}{$\color{labelcolor}\mathrm{\tt (\%o41) }\quad $}
0\mbox{}
\]
%%%%%%%%%%%%%%%

c) Calcula una aproximación de la solución con el método de iteración funcional
tomando $x^{(0)}$ = (0.1, 0.1, -0.1) con una tolerancia fijada de $10^{-5}$, donde la tolerancia viene dada por la norma infinito de la diferencia de dos aproximaciones sucesivas.

d) Sabiendo que la solución del sistema es $x^{*} = \left(0.5, 0, \frac{-\pi}{6}\right)$ calcula el error absoluto cometido en la aproximación obtenida.


e) Calcula, utilizando la cota teórica del método de iteración funcional, el número de iteraciones necesarias para asegurar un error absoluto menor que $10^{-5}$ ¿Qué conclusión extraes?


\end{document}
